\usepackage[most]{tcolorbox}
\usepackage{listings}
\usepackage{textcomp}
\usepackage{xspace}

\newcommand{\originalgrumbler}[2]{\begin{quote}\textcolor{blue}{\sl{\bf #1:} #2}\end{quote}}
\newcommand{\grumbler}[2]{\originalgrumbler{#1}{#2}}
\newcommand{\swarnendu}[1]{\grumbler{Swarnendu}{#1}}
\newcommand{\vipin}[1]{\grumbler{Vipin}{#1}}
\newcommand{\mainak}[1]{\grumbler{Mainak}{#1}}
\newcommand{\binong}[1]{\grumbler{Binong}{#1}}

% For a comment you still want to show up when grumbler is disabled, use \urgent{\yourname{...}}
\newcommand{\urgent}[1]{{\renewcommand{\grumbler}[2]{\originalgrumbler{##1}{##2}}#1}}
\newcommand{\doubt}[1]{\textcolor{violet}{DOUBT: #1}}
\newcommand{\hilite}[1]{\textcolor{red}{HILITE: #1}}
% \newcommand{\remark}[1]{{\bf [ \footnotesize #1 ]}}
\newcommand{\later}[1]{\begin{quote}\textcolor{darkgreen}{\textbackslash \textbf{later\{}} #1 \textcolor{darkgreen}{\}}\end{quote}}
\newcommand{\revise}[1]{\textcolor{orange}{REVISE: #1}}
\newcommand{\savespace}[1]{\textcolor{purple}{SAVESPACE: #1}}

% Define colors
\definecolor{darkgreen}{rgb}{0,0.4,0}
\definecolor{mygreen}{rgb}{0,0.6,0}
\definecolor{mygray}{rgb}{0.5,0.5,0.5}
\definecolor{mauve}{rgb}{0.58,0,0.82}
\definecolor{backgroundColor}{rgb}{0.95,0.95,0.92}

\lstdefinelanguage{Generic}
{
    comment=[l]{//}
}

\lstdefinestyle{GenericStyle}{
    % draw a frame at the top and bottom of the code block
    language=Generic,
    frame=tb,
    backgroundcolor=\color{backgroundColor},
    basicstyle=\small\sffamily,
    keywordstyle=\color{black}\textbf,
    stringstyle=\ttfamily,
    commentstyle=\footnotesize\color{darkgreen},
    % print empty lines at the end of listings
    showlines=true,
    numbers=left,
    numberstyle=\tiny\color{mygray},
    stepnumber=1,
    numbersep=3pt,
    tabsize=2,
    showstringspaces=false
}

\lstdefinestyle{BashStyle}{
    language=bash,
    % draw a frame at the top and bottom of the code block
    frame=tb,
    backgroundcolor=\color{backgroundColor},
    basicstyle=\small\sffamily,
    % print empty lines at the end of listings
    showlines=true,
    numbers=left,
    numberstyle=\tiny\color{mygray},
    stepnumber=1,
    numbersep=3pt,
    tabsize=2,
    showstringspaces=false,
    %columns=fullflexible,
    %linewidth=\textwidth,
    %xleftmargin=0.1\linewidth
    morekeywords={wget, tar, make, sudo}
}

\lstdefinestyle{JavaStyle}{
    language=Java,
    % draw a frame at the top and bottom of the code block
    frame=tb,
    backgroundcolor=\color{backgroundColor},
    basicstyle=\small\sffamily,
    keywordstyle=\color{blue}\textbf,
    commentstyle=\color{darkgreen}\textit,
    stringstyle=\color{mauve}\ttfamily,
    % print empty lines at the end of listings
    showlines=true,
    numbers=left,
    numberstyle=\tiny\color{mygray},
    stepnumber=1,
    numbersep=3pt,
    tabsize=2,
    extendedchars=true,
    breaklines=true,
    escapeinside={\%*}{*)}, % if you want to add LaTeX within your code
    %columns=flexible,
    showstringspaces=false,
    emph={assert},
    emphstyle=\color{blue}\textbf
}

\lstdefinestyle{CPPStyle}{
    language=C++,
    % draw a frame at the top and bottom of the code block
    frame=tb,
    backgroundcolor=\color{backgroundColor},
    basicstyle=\footnotesize\ttfamily,
    keywordstyle=\color{blue}\textbf,
    stringstyle=\color{mauve}\ttfamily,
    commentstyle=\color{darkgreen}\textit,
    morecomment=[l][\color{magenta}]{\#},
    % replaces each occurrence of two consecutive spaces by one,
    % literate = *{\ \ }{\ }1,
    % print empty lines at the end of listings
    showlines=true,
    numbers=left,
    numberstyle=\tiny\color{mygray},
    stepnumber=1,
    numbersep=3pt,
    tabsize=2,
    showstringspaces=false
}

\lstdefinestyle{CStyle}{
    language=C,
    % draw a frame at the top and bottom of the code block
    frame=tb,
    backgroundcolor=\color{backgroundColor},
    basicstyle=\footnotesize\ttfamily,
    keywordstyle=\color{blue}\textbf,
    commentstyle=\color{darkgreen}\textit,
    stringstyle=\color{mauve}\ttfamily,
    morecomment=[l][\color{magenta}]{\#},
    % print empty lines at the end of listings
    showlines=true,
    numbers=left,
    numberstyle=\tiny\color{mygray},
    stepnumber=1,
    numbersep=3pt,
    tabsize=2,
    showstringspaces=false,
    % breakatwhitespace=false,
    %breaklines=true,
    % captionpos=b,
    % keepspaces=true,
    % showspaces=false,
    % showtabs=false,
}

\lstdefinestyle{PythonStyle}{
    language=Python,
    % draw a frame at the top and bottom of the code block
    frame=tb,
    backgroundcolor=\color{backgroundColor},
    basicstyle=\small\ttfamily,
    keywordstyle=\color{blue}\textbf,
    commentstyle=\color{darkgreen}\textit,
    stringstyle=\color{darkgreen}\textit,
    showlines=true, % print empty lines at the end of listings
    numbers=left,
    numberstyle=\tiny\color{mygray},
    stepnumber=1,
    numbersep=3pt,
    tabsize=4,
    otherkeywords={self},             % Add keywords here
    emph={MyClass,__init__},          % Custom highlighting
    emphstyle=\ttb\color{deepred},    % Custom highlighting style
    showstringspaces=false
}

\newcommand{\lstfont}[1]{\color{#1}\scriptsize\ttfamily}

\lstdefinestyle{CUDAStyle}{
    language=C++,
    frame=tb, % draw a frame at the top and bottom of the code block
    backgroundcolor=\color{backgroundColor},
    basicstyle=\footnotesize\ttfamily,
    keywordstyle=\color{blue}\textbf,
    % identifierstyle=\lstfont{white},
    stringstyle=\color{red}\ttfamily,
    commentstyle=\color{darkgreen}\textit,
    emph={
            cudaMalloc, cudaFree, __global__, __shared__, __device__, __host__, __syncthreads, cudaMallocManaged, cudaDeviceSynchronize, cudaMemcpy, cudaStream_t, cudaError_t, cudaMemcpyAsync, cudaStreamCreate, cudaStreamSynchronize, cudaStreamDestroy, __threadfence, cudaGetDevice, cudaMemPrefetchAsync
        },
    emphstyle=\color{darkgreen}\bf\ttfamily,
    % print empty lines at the end of listings
    showlines=true,
    numbers=left,
    numberstyle=\tiny\color{mygray},
    stepnumber=1,
    numbersep=3pt,
    tabsize=2,
    showstringspaces=false,
    % breakatwhitespace=false,
    breaklines=true,
    % captionpos=b,
    % keepspaces=true,
    moredelim=[s][\ttfamily]{<<<}{>>>},
    % showspaces=false,
    % showtabs=false,
}

\lstdefinestyle{TraceStyle}{
    language=C,
    frame=tb,
    backgroundcolor=\color{white},
    commentstyle=\color{mygreen}\textit,
    basicstyle=\footnotesize\ttfamily,
    keywordstyle=\color{magenta},
    numberstyle=\tiny\color{mygray},
    stringstyle=\color{mauve},
    % breakatwhitespace=false,
    breaklines=true,
    showlines=true,
    % captionpos=b,
    % keepspaces=true,
    numbers=left,
    numbersep=3pt,
    % showspaces=false,
    showstringspaces=false,
    % showtabs=false,
    tabsize=2
}

\lstdefinelanguage[RISC-V]{Assembler}
{
    alsoletter={.}, % allow dots in keywords
    alsodigit={0x}, % hex numbers are numbers too!
    morekeywords=[1]{ % instructions
            lb, lh, lw, lbu, lhu,
            sb, sh, sw,
            sll, slli, srl, srli, sra, srai,
            add, addi, sub, lui, auipc,
            xor, xori, or, ori, and, andi,
            slt, slti, sltu, sltiu,
            beq, bne, blt, bge, bltu, bgeu,
            j, jr, jal, jalr, ret,
            scall, break, nop
        },
    morekeywords=[2]{ % sections of our code and other directives
            .align, .ascii, .asciiz, .byte, .data, .double, .extern,
            .float, .globl, .half, .kdata, .ktext, .set, .space, .text, .word
        },
    morekeywords=[3]{ % registers
            zero, ra, sp, gp, tp, s0, fp,
            t0, t1, t2, t3, t4, t5, t6,
            s1, s2, s3, s4, s5, s6, s7, s8, s9, s10, s11,
            a0, a1, a2, a3, a4, a5, a6, a7,
            ft0, ft1, ft2, ft3, ft4, ft5, ft6, ft7,
            fs0, fs1, fs2, fs3, fs4, fs5, fs6, fs7, fs8, fs9, fs10, fs11,
            fa0, fa1, fa2, fa3, fa4, fa5, fa6, fa7
        },
    morecomment=[l]{;},   % mark ; as line comment start
    morecomment=[l]{\#},  % as well as # (even though it is unconventional)
    morestring=[b]",      % mark " as string start/end
    morestring=[b]'       % also mark ' as string start/end
}

\lstdefinelanguage[x64]{Assembler}     % add a "x64" dialect of Assembler
[x86masm]{Assembler} % based on the "x86masm" dialect
{
    % with these extra keywords:
    morekeywords={CDQE,CQO,CMPSQ,CMPXCHG16B,JRCXZ,LODSQ,MOVSXD, %
            POPFQ,PUSHFQ,SCASQ,STOSQ,IRETQ,RDTSCP,SWAPGS, %
            PUSHQ,MOVQ,SUBQ,ADDQ,ADDL,ADDW,MOVL,ADDB, %
            addi, bnez, %
            rax,rdx,rcx,rbx,rsi,rdi,rsp,rbp, %
            r8,r8d,r8w,r8b,r9,r9d,r9w,r9b, %
            r10,r10d,r10w,r10b,r11,r11d,r11w,r11b, %
            r12,r12d,r12w,r12b,r13,r13d,r13w,r13b, %
            r14,r14d,r14w,r14b,r15,r15d,r15w,r15b}, % etc.
    % mark ; as line comment start
    morecomment=[l]{;},
    frame=tb,
    backgroundcolor=\color{backgroundColor},
    basicstyle=\footnotesize\ttfamily,
    keywordstyle=\color{blue},
    stringstyle=\color{mauve}\ttfamily,
    commentstyle=\color{darkgreen}\textit,
    showlines=true,
    numbers=left,
    numberstyle=\tiny\color{mygray},
    stepnumber=1,
    numbersep=3pt
}

\lstdefinestyle{AlgolStyle}{
    language=Algol,
    % draw a frame at the top and bottom of the code block
    frame=tb,
    backgroundcolor=\color{backgroundColor},
    basicstyle=\small\sffamily,
    keywordstyle=\color{black}\textbf,
    commentstyle=\color{darkgreen}\textit,
    stringstyle=\color{mauve}\ttfamily,
    % print empty lines at the end of listings
    showlines=true,
    numbers=left,
    numberstyle=\tiny\color{mygray},
    stepnumber=1,
    numbersep=3pt,
    tabsize=2,
    extendedchars=true,
    breaklines=true,
    escapeinside={\%*}{*)}, % if you want to add LaTeX within your code
    %columns=flexible,
    showstringspaces=false
}

\lstdefinestyle{PascalStyle}{
    language=Pascal,
    % draw a frame at the top and bottom of the code block
    frame=tb,
    backgroundcolor=\color{backgroundColor},
    basicstyle=\small\sffamily,
    keywordstyle=\color{black}\textbf,
    commentstyle=\color{darkgreen}\textit,
    stringstyle=\color{mauve}\ttfamily,
    % print empty lines at the end of listings
    showlines=true,
    numbers=left,
    numberstyle=\tiny\color{mygray},
    stepnumber=1,
    numbersep=3pt,
    tabsize=2,
    extendedchars=true,
    breaklines=true,
    escapeinside={\%*}{*)}, % if you want to add LaTeX within your code
    %columns=flexible,
    showstringspaces=false
}

\lstdefinestyle{PerlStyle}{
    % draw a frame at the top and bottom of the code block
    language=Perl,
    frame=tb,
    backgroundcolor=\color{backgroundColor},
    basicstyle=\small\sffamily,
    keywordstyle=\color{black}\textbf,
    stringstyle=\ttfamily,
    commentstyle=\footnotesize\color{darkgreen},
    % print empty lines at the end of listings
    showlines=true,
    numbers=left,
    numberstyle=\tiny\color{mygray},
    stepnumber=1,
    numbersep=3pt,
    tabsize=2,
    showstringspaces=false
}

\newcommand{\bench}[1]{\textsf{\small #1}}
\newcommand{\code}[1]{\texttt{\small #1}}
\newcommand{\pcode}[1]{\textsf{\small #1}}

\newcommand{\mc}[3]{\multicolumn{#1}{#2}{#3}}
\newcommand{\ra}{\( \rightarrow \)}

\newcommand{\eg}{e.g.\xspace}
\newcommand{\ie}{i.e.\xspace}
\newcommand{\cf}{cf.\xspace}
\newcommand{\etal}{et al.\xspace}
\newcommand{\defacto}{\emph{de facto}\xspace}

% Commonly used words with umlauts
\newcommand{\naive}{na\"{\i}ve\xspace}
\newcommand{\Naive}{Na\"{\i}ve\xspace}
\newcommand{\naively}{na\"{\i}vely\xspace}
\newcommand{\Naively}{Na\"{\i}vely\xspace}

% Suggestion to avoid orphans
\clubpenalty= 10000
\widowpenalty= 10000
\displaywidowpenalty=10000

\tcbset{
    frame code={}
    center title,
    left=0pt,
    right=0pt,
    top=0pt,
    bottom=0pt,
    colback=gray!20,
    colframe=white,
    width=\dimexpr\textwidth\relax,
    enlarge left by=0mm,
    boxsep=5pt,
    arc=0pt,outer arc=0pt,
}

% https://tex.stackexchange.com/questions/4302/prettiest-way-to-typeset-c-cplusplus
\def\CPP{{C\nolinebreak[4]\hspace{-.05em}\raisebox{.1ex}{\footnotesize\bf ++\xspace}}}
\def\CPPNOXSPACE{{C\nolinebreak[4]\hspace{-.05em}\raisebox{.1ex}{\footnotesize\bf ++}}}

\hyphenation{gar-bage}
\hyphenation{program-mer}
\hyphenation{spe-ci-fied}

% \graphicspath{{./figs/}}

\newcommand{\nvidia}{NVIDIA\xspace}
\newcommand{\NVIDIA}{NVIDIA\xspace}

\newcommand{\resneteighteen}{ResNet-18\xspace}
\newcommand{\resnetfifty}{ResNet-50\xspace}
\newcommand{\vggsixteen}{VGG-16\xspace}

% Algorithms
\newcommand{\bigO}[1]{\( \mathcal{O} \left( #1 \right) \)}
\newcommand\letequal{$\gets$\xspace}

\newcommand{\overbar}[1]{\mkern 1.5mu\overline{\mkern-1.5mu#1\mkern-1.5mu}\mkern 1.5mu}
\newcommand{\mathcolorbox}[2]{\colorbox{#1}{$\displaystyle #2$}}

\DeclareMathOperator*{\argmax}{\textsf{arg\,max}}
\DeclareMathOperator*{\argmin}{\textsf{arg\,min}}

% Single quotes in math mode
\DeclareMathSymbol{\mlq}{\mathord}{operators}{``}
\DeclareMathSymbol{\mrq}{\mathord}{operators}{`'}

\newcommand{\cudakernel}[2]{\texttt{\textlangle{}\textlangle{}\textlangle{}}#1,#2\texttt{\textrangle{}\textrangle{}\textrangle{}}}

% Larger-sized bullet
\newcommand\largerbullet[1][.5]{\mathbin{\vcenter{\hbox{\scalebox{#1}{$\bullet$}}}}}

\newcommand*\graymathbox[1]{%
    \colorbox{gray!10}{\hspace{1em}#1\hspace{1em}}%
}

%%%%%%%%%%%%%%%%%%%%%%%%%%%%%%%%%%%%%%%%%%%%%%%%%%%%%%%%%%%%%%%%
% Project-specific

\newcommand{\htuvm}{\textsf{HT-UVM}\xspace}
\newcommand{\Htuvm}{\textsf{HT-UVM}\xspace}

\newcommand{\htovs}{\textsf{HT-OVS}\xspace}
\newcommand{\Htovs}{\textsf{HT-OVS}\xspace}

\newcommand{\skipuvm}{\textsf{SL-UVM}\xspace}
\newcommand{\Skipuvm}{\textsf{SL-UVM}\xspace}

\newcommand{\skipovs}{\textsf{SL-OVS}\xspace}
\newcommand{\Skipovs}{\textsf{SL-OVS}\xspace}

% \newcommand{\heterosl}{\textsf{Hetero-SL}\xspace}
\newcommand{\heterosl}{\textsf{SL-OVS}\xspace}
