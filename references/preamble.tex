%% SB: The following are more or less a basic core set of packages. A few
%% SB: The following are more or less a basic core set of packages. NOTE: A few
%% packages lead to conflicts across different templates. We should avoid
%% enabling such packages globally, they should be included on a per-project
%% basis.

% enables the use of UTF-8 as character encoding
\usepackage[utf8]{inputenc}
% ensures the use of font encodings that support accented characters
\usepackage[T1]{fontenc}
\usepackage{lmodern}
% enables certain features 'towards typographical perfection'
% \usepackage[final]{microtype}
% \usepackage{microtype}
\usepackage{xspace} % Smart spacing after macros
% Relative font sizing, defines \smaller and \larger, but also \relsize{-3} to shrink text thrice.
\usepackage{microtype}
% use this package to get a 2 line header. It is recommended to load fancyhdr after geometry.
% \usepackage{fancyhdr}
\usepackage{color}
\usepackage{xcolor}
\usepackage{etoolbox}

\usepackage{mathtools}

\usepackage{amsmath}
\usepackage{amsfonts}

\usepackage{bm}
\usepackage{textcomp}

%% \usepackage{enumerate}
%% \usepackage[shortlabels]{enumitem}
%% \usepackage{mdwlist}

\usepackage{lscape}
\usepackage{pdflscape}
\usepackage{rotating}

\usepackage{hyperref}
\usepackage{xspace}
% defines \smaller and \larger, but also \relsize{-3} to shrink text thrice.
\usepackage{relsize}
% \usepackage{amsfonts}
% \usepackage{textcomp} % Additional text symbols
% \usepackage{pifont} % Dingbats / special symbols
\usepackage{lmodern} % latin modern fonts

% Colors & Tools

% \usepackage{xcolor} % supersedes color
% \usepackage{etoolbox} % Programming tools for LaTeX
% \usepackage[normalem]{ulem} % Underlining & strikeouts (\sout, etc.)

% Figures & Graphics

% \usepackage{graphicx}
% \usepackage{wrapfig}
% \usepackage{float} % to force figure positioning
% subfigure is deprecated by subfig, but it seems subcaption is now the preferred
% choice.
% \usepackage[caption=false,font=footnotesize]{subfig}
\usepackage{subcaption} % For complex figures with subfigures/subcaptions
% \usepackage{rotating}
% \usepackage{lscape}
% \usepackage{pdflscape}
% \usepackage{tikz}
% \usetikzlibrary{arrows,shapes,decorations,automata,backgrounds,calc,topaths,positioning,matrix}
% \usepackage[framemethod=TikZ]{mdframed} % Framed environments with TikZ

% Math
% \usepackage{amsmath}
% \usepackage{mathtools} % Extends amsmath
% \usepackage{bm} % bold math fonts

% Tables
\usepackage{array}
% \usepackage{booktabs} % For formal tables rules

\usepackage{multirow}
\usepackage{multicol}
% \usepackage{longtable} % Span pages
% \usepackage{colortbl}
\usepackage{booktabs} % For formal tables
\usepackage{longtable}
\usepackage{array}
\usepackage{colortbl}

% Code & Listings
\usepackage{wrapfig}
\usepackage[framemethod=TikZ]{mdframed}
\usepackage{tikz}
\usetikzlibrary{arrows,shapes,decorations,automata,backgrounds,calc,topaths,positioning,matrix}
\usepackage{lipsum}
\usepackage{balance}
%% \usepackage{bookmark}
\usepackage{listings}

% Layout & Balancing
% \usepackage{balance} % Balance last page columns
% \usepackage{flushend}
% use this package to get a 2 line header. It is recommended to load fancyhdr after geometry.
% \usepackage{fancyhdr}

% Algorithms
\usepackage{algorithm} % Used for captioning the algorithm
% IEEE and ACM templates use different packages, choose one of the following two packages to typeset the body
% \usepackage{algorithmicx}
% \usepackage[noend]{algpseudocode}
% \usepackage{algorithmic}

% Footnotes
% \usepackage[multiple]{footmisc}
% \usepackage{tablefootnote}

% Misc
% \usepackage{lipsum}
\usepackage{graphicx}
\usepackage[normalem]{ulem}

\usepackage{pifont}

\usepackage{tablefootnote}
\usepackage[multiple]{footmisc}

\usepackage{float} % to force figure positioning

%% SB: These are too specific and often lead to package conflicts. We should not be enabled these packages globally across all projects.

% https://texdoc.org/serve/natnotes.pdf/0
% \usepackage[square,numbers,sort&compress]{natbib}
% \usepackage[sort]{cite}

%% \usepackage{enumerate}
%% \usepackage[shortlabels]{enumitem}
%% \usepackage{mdwlist}

%% \usepackage{bookmark} % Better bookmarks than hyperref
% SB: subfigure is deprecated by subfig, it seems subcaption is now the preferred choice. We keep both
% commented because otherwise they conflict with each other
% \usepackage[caption=false,font=footnotesize]{subfig}
%% For complex figures with subfigures/subcaptions
\usepackage{subcaption}

% \usepackage[hyphens]{url}
% \usepackage{parskip}

% \usepackage[a4paper,top=2.8cm,left=2cm,right=2cm,bottom=2.5cm]{geometry}
% \usepackage[compact]{titlesec}
% \usepackage{parcolumns}
% \usepackage{charter}
% \usepackage{comment}

% \usepackage{hyperref}
% colorlinks=true, Links will be colored, the default color is red. False will use boxed links.
% linkcolor=blue, Internal links, those generated by cross-referenced elements, are displayed in blue.
% filecolor=magenta, Links to local files will be shown in magenta color (see linking local files).
% urlcolor=cyan, Links to websites are set to cyan color (see linking web addresses).

% \hypersetup{citecolor=black,colorlinks=false,hidelinks=true,linkcolor=blue,filecolor=magenta,urlcolor=cyan}

% SB: This package needs to be loaded after hyperref, so include this package in the main LaTeX file.
% \usepackage{cleveref}

% \usepackage{amssymb}
% \usepackage[expert]{mathdesign}

\usepackage{algorithm} % Used for captioning the algorithm

% https://www.overleaf.com/learn/latex/Algorithms
% IEEE and ACM templates use different packages, choose one of the following two packages to typeset the body
% \usepackage{algorithmicx}
% \usepackage[noend]{algpseudocode}
% \usepackage{algorithmic}

%% SB: Are the following packages popular and useful?

% \usepackage{bbding}
% \usepackage{adjustbox}

\usepackage{flushend}

%%% Local Variables:
%%% mode: LaTeX
%%% TeX-master: "paper"
%%% End:
