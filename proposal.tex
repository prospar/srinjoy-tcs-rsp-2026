\pdfminorversion=7
\documentclass[12pt,a4paper]{article}

%% SB: The following are more or less a basic core set of packages. A few
%% SB: The following are more or less a basic core set of packages. NOTE: A few
%% packages lead to conflicts across different templates. We should avoid
%% enabling such packages globally, they should be included on a per-project
%% basis.

% enables the use of UTF-8 as character encoding
\usepackage[utf8]{inputenc}
% ensures the use of font encodings that support accented characters
\usepackage[T1]{fontenc}
\usepackage{lmodern}
% enables certain features 'towards typographical perfection'
% \usepackage[final]{microtype}
% \usepackage{microtype}
\usepackage{xspace} % Smart spacing after macros
% Relative font sizing, defines \smaller and \larger, but also \relsize{-3} to shrink text thrice.
\usepackage{microtype}
% use this package to get a 2 line header. It is recommended to load fancyhdr after geometry.
% \usepackage{fancyhdr}
\usepackage{color}
\usepackage{xcolor}
\usepackage{etoolbox}

\usepackage{mathtools}

\usepackage{amsmath}
\usepackage{amsfonts}

\usepackage{bm}
\usepackage{textcomp}

%% \usepackage{enumerate}
%% \usepackage[shortlabels]{enumitem}
%% \usepackage{mdwlist}

\usepackage{lscape}
\usepackage{pdflscape}
\usepackage{rotating}

\usepackage{hyperref}
\usepackage{xspace}
% defines \smaller and \larger, but also \relsize{-3} to shrink text thrice.
\usepackage{relsize}
% \usepackage{amsfonts}
% \usepackage{textcomp} % Additional text symbols
% \usepackage{pifont} % Dingbats / special symbols
\usepackage{lmodern} % latin modern fonts

% Colors & Tools

% \usepackage{xcolor} % supersedes color
% \usepackage{etoolbox} % Programming tools for LaTeX
% \usepackage[normalem]{ulem} % Underlining & strikeouts (\sout, etc.)

% Figures & Graphics

% \usepackage{graphicx}
% \usepackage{wrapfig}
% \usepackage{float} % to force figure positioning
% subfigure is deprecated by subfig, but it seems subcaption is now the preferred
% choice.
% \usepackage[caption=false,font=footnotesize]{subfig}
\usepackage{subcaption} % For complex figures with subfigures/subcaptions
% \usepackage{rotating}
% \usepackage{lscape}
% \usepackage{pdflscape}
% \usepackage{tikz}
% \usetikzlibrary{arrows,shapes,decorations,automata,backgrounds,calc,topaths,positioning,matrix}
% \usepackage[framemethod=TikZ]{mdframed} % Framed environments with TikZ

% Math
% \usepackage{amsmath}
% \usepackage{mathtools} % Extends amsmath
% \usepackage{bm} % bold math fonts

% Tables
\usepackage{array}
% \usepackage{booktabs} % For formal tables rules

\usepackage{multirow}
\usepackage{multicol}
% \usepackage{longtable} % Span pages
% \usepackage{colortbl}
\usepackage{booktabs} % For formal tables
\usepackage{longtable}
\usepackage{array}
\usepackage{colortbl}

% Code & Listings
\usepackage{wrapfig}
\usepackage[framemethod=TikZ]{mdframed}
\usepackage{tikz}
\usetikzlibrary{arrows,shapes,decorations,automata,backgrounds,calc,topaths,positioning,matrix}
\usepackage{lipsum}
\usepackage{balance}
%% \usepackage{bookmark}
\usepackage{listings}

% Layout & Balancing
% \usepackage{balance} % Balance last page columns
% \usepackage{flushend}
% use this package to get a 2 line header. It is recommended to load fancyhdr after geometry.
% \usepackage{fancyhdr}

% Algorithms
\usepackage{algorithm} % Used for captioning the algorithm
% IEEE and ACM templates use different packages, choose one of the following two packages to typeset the body
% \usepackage{algorithmicx}
% \usepackage[noend]{algpseudocode}
% \usepackage{algorithmic}

% Footnotes
% \usepackage[multiple]{footmisc}
% \usepackage{tablefootnote}

% Misc
% \usepackage{lipsum}
\usepackage{graphicx}
\usepackage[normalem]{ulem}

\usepackage{pifont}

\usepackage{tablefootnote}
\usepackage[multiple]{footmisc}

\usepackage{float} % to force figure positioning

%% SB: These are too specific and often lead to package conflicts. We should not be enabled these packages globally across all projects.

% https://texdoc.org/serve/natnotes.pdf/0
% \usepackage[square,numbers,sort&compress]{natbib}
% \usepackage[sort]{cite}

%% \usepackage{enumerate}
%% \usepackage[shortlabels]{enumitem}
%% \usepackage{mdwlist}

%% \usepackage{bookmark} % Better bookmarks than hyperref
% SB: subfigure is deprecated by subfig, it seems subcaption is now the preferred choice. We keep both
% commented because otherwise they conflict with each other
% \usepackage[caption=false,font=footnotesize]{subfig}
%% For complex figures with subfigures/subcaptions
\usepackage{subcaption}

% \usepackage[hyphens]{url}
% \usepackage{parskip}

% \usepackage[a4paper,top=2.8cm,left=2cm,right=2cm,bottom=2.5cm]{geometry}
% \usepackage[compact]{titlesec}
% \usepackage{parcolumns}
% \usepackage{charter}
% \usepackage{comment}

% \usepackage{hyperref}
% colorlinks=true, Links will be colored, the default color is red. False will use boxed links.
% linkcolor=blue, Internal links, those generated by cross-referenced elements, are displayed in blue.
% filecolor=magenta, Links to local files will be shown in magenta color (see linking local files).
% urlcolor=cyan, Links to websites are set to cyan color (see linking web addresses).

% \hypersetup{citecolor=black,colorlinks=false,hidelinks=true,linkcolor=blue,filecolor=magenta,urlcolor=cyan}

% SB: This package needs to be loaded after hyperref, so include this package in the main LaTeX file.
% \usepackage{cleveref}

% \usepackage{amssymb}
% \usepackage[expert]{mathdesign}

\usepackage{algorithm} % Used for captioning the algorithm

% https://www.overleaf.com/learn/latex/Algorithms
% IEEE and ACM templates use different packages, choose one of the following two packages to typeset the body
% \usepackage{algorithmicx}
% \usepackage[noend]{algpseudocode}
% \usepackage{algorithmic}

%% SB: Are the following packages popular and useful?

% \usepackage{bbding}
% \usepackage{adjustbox}

\usepackage{flushend}

%%% Local Variables:
%%% mode: LaTeX
%%% TeX-master: "paper"
%%% End:

\usepackage[most]{tcolorbox}
\usepackage{listings}
\usepackage{textcomp}
\usepackage{xspace}

\newcommand{\originalgrumbler}[2]{\begin{quote}\textcolor{blue}{\sl{\bf #1:} #2}\end{quote}}
\newcommand{\grumbler}[2]{\originalgrumbler{#1}{#2}}
\newcommand{\swarnendu}[1]{\grumbler{Swarnendu}{#1}}
\newcommand{\vipin}[1]{\grumbler{Vipin}{#1}}
\newcommand{\mainak}[1]{\grumbler{Mainak}{#1}}
\newcommand{\binong}[1]{\grumbler{Binong}{#1}}

% For a comment you still want to show up when grumbler is disabled, use \urgent{\yourname{...}}
\newcommand{\urgent}[1]{{\renewcommand{\grumbler}[2]{\originalgrumbler{##1}{##2}}#1}}
\newcommand{\doubt}[1]{\textcolor{violet}{DOUBT: #1}}
\newcommand{\hilite}[1]{\textcolor{red}{HILITE: #1}}
% \newcommand{\remark}[1]{{\bf [ \footnotesize #1 ]}}
\newcommand{\later}[1]{\begin{quote}\textcolor{darkgreen}{\textbackslash \textbf{later\{}} #1 \textcolor{darkgreen}{\}}\end{quote}}
\newcommand{\revise}[1]{\textcolor{orange}{REVISE: #1}}
\newcommand{\savespace}[1]{\textcolor{purple}{SAVESPACE: #1}}

% Define colors
\definecolor{darkgreen}{rgb}{0,0.4,0}
\definecolor{mygreen}{rgb}{0,0.6,0}
\definecolor{mygray}{rgb}{0.5,0.5,0.5}
\definecolor{mauve}{rgb}{0.58,0,0.82}
\definecolor{backgroundColor}{rgb}{0.95,0.95,0.92}

\lstdefinelanguage{Generic}
{
    comment=[l]{//}
}

\lstdefinestyle{GenericStyle}{
    % draw a frame at the top and bottom of the code block
    language=Generic,
    frame=tb,
    backgroundcolor=\color{backgroundColor},
    basicstyle=\small\sffamily,
    keywordstyle=\color{black}\textbf,
    stringstyle=\ttfamily,
    commentstyle=\footnotesize\color{darkgreen},
    % print empty lines at the end of listings
    showlines=true,
    numbers=left,
    numberstyle=\tiny\color{mygray},
    stepnumber=1,
    numbersep=3pt,
    tabsize=2,
    showstringspaces=false
}

\lstdefinestyle{BashStyle}{
    language=bash,
    % draw a frame at the top and bottom of the code block
    frame=tb,
    backgroundcolor=\color{backgroundColor},
    basicstyle=\small\sffamily,
    % print empty lines at the end of listings
    showlines=true,
    numbers=left,
    numberstyle=\tiny\color{mygray},
    stepnumber=1,
    numbersep=3pt,
    tabsize=2,
    showstringspaces=false,
    %columns=fullflexible,
    %linewidth=\textwidth,
    %xleftmargin=0.1\linewidth
    morekeywords={wget, tar, make, sudo}
}

\lstdefinestyle{JavaStyle}{
    language=Java,
    % draw a frame at the top and bottom of the code block
    frame=tb,
    backgroundcolor=\color{backgroundColor},
    basicstyle=\small\sffamily,
    keywordstyle=\color{blue}\textbf,
    commentstyle=\color{darkgreen}\textit,
    stringstyle=\color{mauve}\ttfamily,
    % print empty lines at the end of listings
    showlines=true,
    numbers=left,
    numberstyle=\tiny\color{mygray},
    stepnumber=1,
    numbersep=3pt,
    tabsize=2,
    extendedchars=true,
    breaklines=true,
    escapeinside={\%*}{*)}, % if you want to add LaTeX within your code
    %columns=flexible,
    showstringspaces=false,
    emph={assert},
    emphstyle=\color{blue}\textbf
}

\lstdefinestyle{CPPStyle}{
    language=C++,
    % draw a frame at the top and bottom of the code block
    frame=tb,
    backgroundcolor=\color{backgroundColor},
    basicstyle=\footnotesize\ttfamily,
    keywordstyle=\color{blue}\textbf,
    stringstyle=\color{mauve}\ttfamily,
    commentstyle=\color{darkgreen}\textit,
    morecomment=[l][\color{magenta}]{\#},
    % replaces each occurrence of two consecutive spaces by one,
    % literate = *{\ \ }{\ }1,
    % print empty lines at the end of listings
    showlines=true,
    numbers=left,
    numberstyle=\tiny\color{mygray},
    stepnumber=1,
    numbersep=3pt,
    tabsize=2,
    showstringspaces=false
}

\lstdefinestyle{CStyle}{
    language=C,
    % draw a frame at the top and bottom of the code block
    frame=tb,
    backgroundcolor=\color{backgroundColor},
    basicstyle=\footnotesize\ttfamily,
    keywordstyle=\color{blue}\textbf,
    commentstyle=\color{darkgreen}\textit,
    stringstyle=\color{mauve}\ttfamily,
    morecomment=[l][\color{magenta}]{\#},
    % print empty lines at the end of listings
    showlines=true,
    numbers=left,
    numberstyle=\tiny\color{mygray},
    stepnumber=1,
    numbersep=3pt,
    tabsize=2,
    showstringspaces=false,
    % breakatwhitespace=false,
    %breaklines=true,
    % captionpos=b,
    % keepspaces=true,
    % showspaces=false,
    % showtabs=false,
}

\lstdefinestyle{PythonStyle}{
    language=Python,
    % draw a frame at the top and bottom of the code block
    frame=tb,
    backgroundcolor=\color{backgroundColor},
    basicstyle=\small\ttfamily,
    keywordstyle=\color{blue}\textbf,
    commentstyle=\color{darkgreen}\textit,
    stringstyle=\color{darkgreen}\textit,
    showlines=true, % print empty lines at the end of listings
    numbers=left,
    numberstyle=\tiny\color{mygray},
    stepnumber=1,
    numbersep=3pt,
    tabsize=4,
    otherkeywords={self},             % Add keywords here
    emph={MyClass,__init__},          % Custom highlighting
    emphstyle=\ttb\color{deepred},    % Custom highlighting style
    showstringspaces=false
}

\newcommand{\lstfont}[1]{\color{#1}\scriptsize\ttfamily}

\lstdefinestyle{CUDAStyle}{
    language=C++,
    frame=tb, % draw a frame at the top and bottom of the code block
    backgroundcolor=\color{backgroundColor},
    basicstyle=\footnotesize\ttfamily,
    keywordstyle=\color{blue}\textbf,
    % identifierstyle=\lstfont{white},
    stringstyle=\color{red}\ttfamily,
    commentstyle=\color{darkgreen}\textit,
    emph={
            cudaMalloc, cudaFree, __global__, __shared__, __device__, __host__, __syncthreads, cudaMallocManaged, cudaDeviceSynchronize, cudaMemcpy, cudaStream_t, cudaError_t, cudaMemcpyAsync, cudaStreamCreate, cudaStreamSynchronize, cudaStreamDestroy, __threadfence, cudaGetDevice, cudaMemPrefetchAsync
        },
    emphstyle=\color{darkgreen}\bf\ttfamily,
    % print empty lines at the end of listings
    showlines=true,
    numbers=left,
    numberstyle=\tiny\color{mygray},
    stepnumber=1,
    numbersep=3pt,
    tabsize=2,
    showstringspaces=false,
    % breakatwhitespace=false,
    breaklines=true,
    % captionpos=b,
    % keepspaces=true,
    moredelim=[s][\ttfamily]{<<<}{>>>},
    % showspaces=false,
    % showtabs=false,
}

\lstdefinestyle{TraceStyle}{
    language=C,
    frame=tb,
    backgroundcolor=\color{white},
    commentstyle=\color{mygreen}\textit,
    basicstyle=\footnotesize\ttfamily,
    keywordstyle=\color{magenta},
    numberstyle=\tiny\color{mygray},
    stringstyle=\color{mauve},
    % breakatwhitespace=false,
    breaklines=true,
    showlines=true,
    % captionpos=b,
    % keepspaces=true,
    numbers=left,
    numbersep=3pt,
    % showspaces=false,
    showstringspaces=false,
    % showtabs=false,
    tabsize=2
}

\lstdefinelanguage[RISC-V]{Assembler}
{
    alsoletter={.}, % allow dots in keywords
    alsodigit={0x}, % hex numbers are numbers too!
    morekeywords=[1]{ % instructions
            lb, lh, lw, lbu, lhu,
            sb, sh, sw,
            sll, slli, srl, srli, sra, srai,
            add, addi, sub, lui, auipc,
            xor, xori, or, ori, and, andi,
            slt, slti, sltu, sltiu,
            beq, bne, blt, bge, bltu, bgeu,
            j, jr, jal, jalr, ret,
            scall, break, nop
        },
    morekeywords=[2]{ % sections of our code and other directives
            .align, .ascii, .asciiz, .byte, .data, .double, .extern,
            .float, .globl, .half, .kdata, .ktext, .set, .space, .text, .word
        },
    morekeywords=[3]{ % registers
            zero, ra, sp, gp, tp, s0, fp,
            t0, t1, t2, t3, t4, t5, t6,
            s1, s2, s3, s4, s5, s6, s7, s8, s9, s10, s11,
            a0, a1, a2, a3, a4, a5, a6, a7,
            ft0, ft1, ft2, ft3, ft4, ft5, ft6, ft7,
            fs0, fs1, fs2, fs3, fs4, fs5, fs6, fs7, fs8, fs9, fs10, fs11,
            fa0, fa1, fa2, fa3, fa4, fa5, fa6, fa7
        },
    morecomment=[l]{;},   % mark ; as line comment start
    morecomment=[l]{\#},  % as well as # (even though it is unconventional)
    morestring=[b]",      % mark " as string start/end
    morestring=[b]'       % also mark ' as string start/end
}

\lstdefinelanguage[x64]{Assembler}     % add a "x64" dialect of Assembler
[x86masm]{Assembler} % based on the "x86masm" dialect
{
    % with these extra keywords:
    morekeywords={CDQE,CQO,CMPSQ,CMPXCHG16B,JRCXZ,LODSQ,MOVSXD, %
            POPFQ,PUSHFQ,SCASQ,STOSQ,IRETQ,RDTSCP,SWAPGS, %
            PUSHQ,MOVQ,SUBQ,ADDQ,ADDL,ADDW,MOVL,ADDB, %
            addi, bnez, %
            rax,rdx,rcx,rbx,rsi,rdi,rsp,rbp, %
            r8,r8d,r8w,r8b,r9,r9d,r9w,r9b, %
            r10,r10d,r10w,r10b,r11,r11d,r11w,r11b, %
            r12,r12d,r12w,r12b,r13,r13d,r13w,r13b, %
            r14,r14d,r14w,r14b,r15,r15d,r15w,r15b}, % etc.
    % mark ; as line comment start
    morecomment=[l]{;},
    frame=tb,
    backgroundcolor=\color{backgroundColor},
    basicstyle=\footnotesize\ttfamily,
    keywordstyle=\color{blue},
    stringstyle=\color{mauve}\ttfamily,
    commentstyle=\color{darkgreen}\textit,
    showlines=true,
    numbers=left,
    numberstyle=\tiny\color{mygray},
    stepnumber=1,
    numbersep=3pt
}

\lstdefinestyle{AlgolStyle}{
    language=Algol,
    % draw a frame at the top and bottom of the code block
    frame=tb,
    backgroundcolor=\color{backgroundColor},
    basicstyle=\small\sffamily,
    keywordstyle=\color{black}\textbf,
    commentstyle=\color{darkgreen}\textit,
    stringstyle=\color{mauve}\ttfamily,
    % print empty lines at the end of listings
    showlines=true,
    numbers=left,
    numberstyle=\tiny\color{mygray},
    stepnumber=1,
    numbersep=3pt,
    tabsize=2,
    extendedchars=true,
    breaklines=true,
    escapeinside={\%*}{*)}, % if you want to add LaTeX within your code
    %columns=flexible,
    showstringspaces=false
}

\lstdefinestyle{PascalStyle}{
    language=Pascal,
    % draw a frame at the top and bottom of the code block
    frame=tb,
    backgroundcolor=\color{backgroundColor},
    basicstyle=\small\sffamily,
    keywordstyle=\color{black}\textbf,
    commentstyle=\color{darkgreen}\textit,
    stringstyle=\color{mauve}\ttfamily,
    % print empty lines at the end of listings
    showlines=true,
    numbers=left,
    numberstyle=\tiny\color{mygray},
    stepnumber=1,
    numbersep=3pt,
    tabsize=2,
    extendedchars=true,
    breaklines=true,
    escapeinside={\%*}{*)}, % if you want to add LaTeX within your code
    %columns=flexible,
    showstringspaces=false
}

\lstdefinestyle{PerlStyle}{
    % draw a frame at the top and bottom of the code block
    language=Perl,
    frame=tb,
    backgroundcolor=\color{backgroundColor},
    basicstyle=\small\sffamily,
    keywordstyle=\color{black}\textbf,
    stringstyle=\ttfamily,
    commentstyle=\footnotesize\color{darkgreen},
    % print empty lines at the end of listings
    showlines=true,
    numbers=left,
    numberstyle=\tiny\color{mygray},
    stepnumber=1,
    numbersep=3pt,
    tabsize=2,
    showstringspaces=false
}

\newcommand{\bench}[1]{\textsf{\small #1}}
\newcommand{\code}[1]{\texttt{\small #1}}
\newcommand{\pcode}[1]{\textsf{\small #1}}

\newcommand{\mc}[3]{\multicolumn{#1}{#2}{#3}}
\newcommand{\ra}{\( \rightarrow \)}

\newcommand{\eg}{e.g.\xspace}
\newcommand{\ie}{i.e.\xspace}
\newcommand{\cf}{cf.\xspace}
\newcommand{\etal}{et al.\xspace}
\newcommand{\defacto}{\emph{de facto}\xspace}

% Commonly used words with umlauts
\newcommand{\naive}{na\"{\i}ve\xspace}
\newcommand{\Naive}{Na\"{\i}ve\xspace}
\newcommand{\naively}{na\"{\i}vely\xspace}
\newcommand{\Naively}{Na\"{\i}vely\xspace}

% Suggestion to avoid orphans
\clubpenalty= 10000
\widowpenalty= 10000
\displaywidowpenalty=10000

\tcbset{
    frame code={}
    center title,
    left=0pt,
    right=0pt,
    top=0pt,
    bottom=0pt,
    colback=gray!20,
    colframe=white,
    width=\dimexpr\textwidth\relax,
    enlarge left by=0mm,
    boxsep=5pt,
    arc=0pt,outer arc=0pt,
}

% https://tex.stackexchange.com/questions/4302/prettiest-way-to-typeset-c-cplusplus
\def\CPP{{C\nolinebreak[4]\hspace{-.05em}\raisebox{.1ex}{\footnotesize\bf ++\xspace}}}
\def\CPPNOXSPACE{{C\nolinebreak[4]\hspace{-.05em}\raisebox{.1ex}{\footnotesize\bf ++}}}

\hyphenation{gar-bage}
\hyphenation{program-mer}
\hyphenation{spe-ci-fied}

% \graphicspath{{./figs/}}

\newcommand{\nvidia}{NVIDIA\xspace}
\newcommand{\NVIDIA}{NVIDIA\xspace}

\newcommand{\resneteighteen}{ResNet-18\xspace}
\newcommand{\resnetfifty}{ResNet-50\xspace}
\newcommand{\vggsixteen}{VGG-16\xspace}

% Algorithms
\newcommand{\bigO}[1]{\( \mathcal{O} \left( #1 \right) \)}
\newcommand\letequal{$\gets$\xspace}

\newcommand{\overbar}[1]{\mkern 1.5mu\overline{\mkern-1.5mu#1\mkern-1.5mu}\mkern 1.5mu}
\newcommand{\mathcolorbox}[2]{\colorbox{#1}{$\displaystyle #2$}}

\DeclareMathOperator*{\argmax}{\textsf{arg\,max}}
\DeclareMathOperator*{\argmin}{\textsf{arg\,min}}

% Single quotes in math mode
\DeclareMathSymbol{\mlq}{\mathord}{operators}{``}
\DeclareMathSymbol{\mrq}{\mathord}{operators}{`'}

\newcommand{\cudakernel}[2]{\texttt{\textlangle{}\textlangle{}\textlangle{}}#1,#2\texttt{\textrangle{}\textrangle{}\textrangle{}}}

% Larger-sized bullet
\newcommand\largerbullet[1][.5]{\mathbin{\vcenter{\hbox{\scalebox{#1}{$\bullet$}}}}}

\newcommand*\graymathbox[1]{%
    \colorbox{gray!10}{\hspace{1em}#1\hspace{1em}}%
}

%%%%%%%%%%%%%%%%%%%%%%%%%%%%%%%%%%%%%%%%%%%%%%%%%%%%%%%%%%%%%%%%
% Project-specific

\newcommand{\htuvm}{\textsf{HT-UVM}\xspace}
\newcommand{\Htuvm}{\textsf{HT-UVM}\xspace}

\newcommand{\htovs}{\textsf{HT-OVS}\xspace}
\newcommand{\Htovs}{\textsf{HT-OVS}\xspace}

\newcommand{\skipuvm}{\textsf{SL-UVM}\xspace}
\newcommand{\Skipuvm}{\textsf{SL-UVM}\xspace}

\newcommand{\skipovs}{\textsf{SL-OVS}\xspace}
\newcommand{\Skipovs}{\textsf{SL-OVS}\xspace}

% \newcommand{\heterosl}{\textsf{Hetero-SL}\xspace}
\newcommand{\heterosl}{\textsf{SL-OVS}\xspace}


\usepackage{titling}
\usepackage[compact]{titlesec}
\usepackage[a4paper,top=0.8in,left=0.8in,right=0.8in,bottom=0.8in]{geometry}
\usepackage{cite}

\usepackage{fancyhdr}  % use this package to get a 2 line header
\pagestyle{fancy}
\fancyhf{}
\fancyhead[L]{\it TCS RSP 2026\/}
\fancyhead[R]{{\bf Srinjoy Sarkar}}
\fancyfoot{}
\fancyfoot[R]{\thepage}

\renewcommand{\headrulewidth}{1.0pt}
\renewcommand{\footrulewidth}{0.4pt}
\setlength{\headheight}{18pt}

\usepackage{hyperref}
\hypersetup{
  breaklinks, % Allow links to break across lines
  hidelinks, % Set to true to hide colored boxes and use plain text links
  urlcolor=black, % Color for URLs
  % Border color if boxes are used (only if colorlinks=false)
  linkbordercolor=white,
  % Links will be colored, the default color is red. False will use boxed links.
  colorlinks=true,
  % Internal links, those generated by cross-referenced elements, are
  % displayed in blue.
  linkcolor=blue,
  % Links to local files will be shown in magenta color (see linking local files)
  filecolor=magenta,
  pdftitle={}, pdfauthor={Swarnendu Biswas}, pdfsubject={}
}

\setlength{\droptitle}{-2cm}
\title{Designing Efficient GPU Data Structures for Oversubscription}
\author{Srinjoy Sarkar \\
    Department of Computer Science Engineering, \\
Indian Institute of Technology Kanpur, India,  \\
\texttt{srinjoys23@cse.iitk.ac.in}}

\date{} %leave this blank

\begin{document}
\maketitle
\thispagestyle{fancy}

\begin{abstract}
    Efficient concurrent data structures are important building blocks for
    accelerating applications on GPUs. With ever-increasing memory footprint of
    GPU workloads, data structures used by kernels can exceed global memory
    capacity. Using the unified virtual memory (UVM) model is a popular approach
    for kernels to oversubscribe GPU memory without the need for explicit memory
    management by a programmer. However, data structures executing with UVM can
    suffer from performance degradation due to the high overheads associated with
    data migration and thrashing under irregular access patterns.

    We propose to work on designing efficient GPU data structure to handle use
    cases where the application working set oversubscribes GPU memory. In
    particular, we will focus on the following data structures popularly used in
    GPU kernels: hash table, skip list, and approximate membership filter. We have
    profiled existing in-memory implementations of these data structures, and have
    identified several bottlenecks. These data structures are characterized by
    random accesses, and hence encounter frequent pages faults and unnecessary
    data migrations between the CPU and the GPU with the UVM programming model. We
    plan to explore alternate designs (\eg, hierarchical layout) for these data
    structures to maximize access locality and hence improve performance. The
    insight is that the outer-level containers enable efficient jumps to desired
    regions of the data structure, while the inner containers allows operating on
    the data. Our preliminary results show that our custom data structure designs
    substantially improve performance over the optimized UVM baselines while
    supporting high degree of GPU memory oversubscription for many inputs.
\end{abstract}

\section*{Background and Motivation}

The high compute capability and memory bandwidth of GPUs, coupled with energy efficiency, have spurred their usage across a diverse set of application domains including scientific computing simulations, big data and large-scale graph analytics, and computational biology~\cite{nvidia-applications}. Improved compute capabilities have led to a steady rise in the working set sizes of
GPU-accelerated applications, which now often far exceed the GPU global memory capacity~\cite{dynamap,uvm-forest-isca25}.
Hence, the traditional copy-then-execute (CTE) programming model is giving way to the newer unified virtual memory (UVM) paradigm, where the CPU virtual address space is also visible to GPU kernels. GPU vendors have introduced memory virtualization support through features such
as unified virtual addressing, demand paging, and prefetching. UVM allows the host CPU and discrete GPU devices to share a unified virtual address space, enabling the GPU runtime to page memory in and out as needed~\cite{hum-ppopp-2020,dynamap}. Demand paging in UVM-capable GPUs facilitates
transparent memory oversubscription, improving programmer productivity. The cost of servicing GPU page faults can be reduced by prefetching a chunk of nearby pages~\cite{cppe-ipdps-2020,uvmsmart-isca-2019,uvm-forest-isca25,hpe-tcad-2020}.

Writing correct but scalable multithreaded GPU kernels is a difficult art~\cite{massively-parallel-processors,kernelbench-arxiv}. In addition to the algorithm and the compute capabilities of the GPU device, the scalability of the application depends on the building blocks of the kernel, \eg, GPU data structures. For example, the performance of the Pinterest application improved substantially by switching to GPU-accelerated hash tables~\cite{pinterest-perf-boost}. Therefore, the design of sophisticated GPU data structures such as lists and trees is an active area of research~\cite{gpu-gems2}. Much recent work has focused on designing efficient data structures (\eg, hash tables, skip lists, B-trees, and membership query filters) for use in device code~\cite{slabhash-ipdps-2020,dachash-sbacpad-2021,warpcore-hipc-2020,warpdrive,parallel-hashmap-nvidia,survey-hashing-tpds-2020,bght,dycuckoo,gph-pamd-2025,gfsl-pact-2017,cmedia-skiplist-ipdpsw-2019,skiplist-survey-2025,multiversion-btree,gpu-filters-ppopp23,harmonia-gpu,gpu-btree,gpu-gelhash,survey-hashing-tpds-2020,lockfree-ds-icpads-2012,quotient-filter,stdgpu,cucollections-nvidia,gpu-cds}.

\paragraph*{The Problem}

The steady increase in the working set size and irregular access patterns of GPU applications require more efficient memory management so that the GPU data structures efficiently scale with the input size. However, existing studies primarily focus on the traditional CTE model (\ie, in-memory data structures) and do not address how to design efficient data structures that oversubscribe memory, and developers fall back to UVM. While UVM simplifies programming, its performance remains a significant concern due to frequent far faults
\footnote{Far faults, which occur when a requested page is absent in GPU global memory, require multiple PCIe roundtrips, and are resolved after interacting with the CPU page tables. On \NVIDIA GPUs, a far fault from a CUDA thread stalls the progress of the warp, possibly affecting the overall GPU throughput.},
high address translation overheads, and high volume of page replacement~\cite{suv-micro-2024,in-depth-analysis-uvm-sc21,fine-grain-quantitative-uvm-taco24,batch-aware-asplos-2020,uvmsmart-isca-2019,cppe-ipdps-2020,irregular-page-walks-isca-2018,heterogeneous-uvm,uvm-forest-isca25,dynamap,uvmbench-arxiv}. For instance, Lin et al.~\cite{oversubscription-dl-training} find UVM-based implementations competitive with the CTE model for workloads that fit in GPU memory, but about $7\times$ slower with oversubscription. The performance degradation with UVM is more prominent in applications with irregular access patterns and pointer chasing~\cite{fine-grain-quantitative-uvm-taco24,demystifying-uvm-cost-ipdps2021,in-depth-analysis-uvm-sc21}.

Hash tables is a popular concurrent data structures for storing key-value
pairs~\cite{art-of-multiprocessor-programming}. Hash tables are widely used in
various scientific applications like metagenomic
sequencing~\cite{metacache-icpp-2021}, photodna~\cite{photodna-ares-2023}, and
database indexing workloads that are accelerated using the GPU but oversubscribe
GPU memory. In metagenomic sequencing, a hash table is used as the genome
database that is later used for query processing. The size of the entire dataset
can grow to 50+~GB, easily overflowing the GPU capacity. Skip lists are
popularly used in
databases~\cite{rocksdb-meta-2012,leveldb-google-2011,pebblesdb-sosp-2017},
networking~\cite{deterministic-skiplists-soda-1992}, persistent
memory~\cite{listdb-osdi-2022,phast-tpds-2022}, concurrency
control~\cite{kiwi-ppopp-2017,nitro-vldbe-2016,jellyfish-middleware-2020}, and
for representing skewed data access
patterns~\cite{biased-skiplist-algorithm-2001,biased-skiplist-algorithmica-2005}.

Data structures like hash tables, skip lists, and AMQs are characterized by input-dependent access locality~\cite{art-of-multiprocessor-programming-v2}.
% As a result,
Lack of locality in the input operation strings and sub-optimal implementation
choices can lead to irregular access patterns and poor data locality. When
memory oversubscription is coupled with these performance pathologies, these
data structures exhibit very poor scalability. In this proposal, we establish
that UVM-based implementations of popular random-access data structures with GPU
memory oversubscription do not scale well with input size even when these
implementations are meticulously optimized using data usage hints and prefetches
based on memory/translation usage~\cite{nvidia-prefetching}. We observe that the
run times of these implementations increase exponentially with a linear increase
in the input sizes indicating a clear \emph{performance bottleneck}. We squarely
focus on alleviating this performance pathology by proposing efficient designs
of concurrent data structures for GPUs.

\section*{Brief Literature Review}

In this section, we briefly review existing approaches for designing hash tables and skip lists for GPUs without memory oversubscription. We are not aware of existing work that focuses on improving performance for UVM-enabled GPU applications.

\paragraph*{Hash Table}

Hash tables are popularly used for tracking key-value relationships in a
sparse domain because of their expected constant time complexity for
different operations like insert and search. The performance of a hash
table depends on the memory layout, access pattern dictated by the input
operation string, quality of the hash functions, and other design decisions
like collision resolution and dynamic resizing techniques. Different
approaches [25, 40] have been explored for CPUs over several decades and
many libraries provide efficient CPU implementations with different features
(e.g., concurrent hash maps from \textsf{oneTBB}\footnote{\url{https://github.com/uxlfoundation/oneTBB}} and \textsf{Folly}\footnote{\url{https://github.com/facebook/folly}}).

Hash table designs for GPUs can be divided into static~\cite{cudpp-hashtable,gpu-stadium-hashing,gpu-parallel-hashing,lockfree-ds-icpads-2012,warpcore-hipc-2020,hashgraph} and dynamic~\cite{slabhash-ipdps-2020,dachash-sbacpad-2021,warpdrive,bght,gph-pamd-2025,dycuckoo} designs. The container array in the
hash table is not resizable in the static design, which implies that the
former category of approaches make a worst-case estimate on the number of
unique keys before allocating memory. Static memory allocation without
knowing an accurate estimate of the number of unique keys can waste memory.

GPU data structure kernels provide bulk APIs that work on multiple data
elements in parallel to avoid frequent expensive kernel launches~\cite{slabhash-ipdps-2020,warpcore-hipc-2020,gfsl-pact-2017,gph-pamd-2025}. A natural division of work in bulk-API- based kernels is to assign an
operation on a (key, value) pair to a CUDA thread. However, given the
potentially irregular input-dependent access pattern of the data structures,
assigning a different key to each thread in a warp leads to branch and
memory divergence. The Warp-Cooperative Work-Sharing (WCWS) strategy,
popularly followed by recent approaches~\cite{slabhash-ipdps-2020,warpcore-hipc-2020,warpdrive,gfsl-pact-2017,gpu-stadium-hashing}, tries to ameliorate
branch and memory divergence by assigning an operation to a warp.

\emph{SlabHash}~\cite{slabhash-ipdps-2020} uses closed addressing for conflict
resolution. Unlike traditional linked lists, each bucket in SlabHash is
organized as a list of super-nodes where each super-node consists of multiple
key-value pairs including a pointer to the next node (Figure 1). A super-node is
sized based on the warp size (e.g., 32 for NVIDIA GPUs). SlabHash requires a
custom memory allocator to handle frequent and concurrent dynamic allocation of
super-nodes to support resizing. \emph{DACHash}~\cite{dachash-sbacpad-2021}
extends SlabHash to sort the input sequence for better access locality, and
dynamically assigns work to either each thread or a warp based on the number of
super-nodes in the relevant bucket. \emph{WarpCore}~\cite{warpcore-hipc-2020}
proposes a static hash table, i.e., the table is not dynamically resized, and
uses open addressing for resolving collisions. WarpCore assigns an operation on
a (key, value) pair to one cooperative group, and all threads in a cooperative
group participate to carry out the operation (e.g., performs a parallel probe to
resolve conflicts). WarpCore may potentially suffer from high overheads in case
there is a large proportion of negative queries (e.g., failed delete or
search operations) at high load factors because it has to keep iterating
through the hash table till it encounters an empty slot~\cite{bght}.

\emph{DyCuckoo}~\cite{dycuckoo} divides the hash table array into d smaller
tables of equal capacity and uses Cuckoo hashing~\cite{cuckoo-hashing} to lower
the cost of rehashing when the table is resized.
% The hash table is downsized or upsized if the load factor crosses respective
% thresholds. Thus, DyCuckoo improves the amortized cost of hash table
% operations that includes negative queries and dynamic resizing.
GPH~\cite{gph-pamd-2025} improves the performance of searches by adapting a
perfect hashing scheme. It uses a cost model to gauge the performance of the
hash table based on the occupancy, number of instructions executed, and the
cycles per instruction. GPH divides the hash table into $n$ contiguous buckets,
and maintains an auxiliary array in per-SM shared memory to store the mapping
and offset for each bucket. GPH uses three hash functions to calculate the
destination bucket for a key $k$.
% The first hash function maps the key to an auxiliary array index I . The
% second hash function calculates the virtual bucket ID v using the mapping
% stored at index I and key k. The third hash function computes the destination
% bucket.
The auxiliary array allows GPH to perform insert and search operations with a
fixed number of probes.

\begin{figure}[t]
    \centering
    \includegraphics[width=0.8\columnwidth]{"./figs/basic-skiplist-design-simplified.drawio.pdf"}
    \caption{Layout of a traditional skip list. The data layer shows the keys in stored in the skip list. The dotted bold black line shows the sequence of pointers traversed while searching for the key 11.}
    \label{fig:skiplist-design}
\end{figure}

\paragraph*{Skip List}

A skip list~\cite{skiplist-cacm-1990} is a probabilistic data structure that
allows efficient insertion, deletion, and search operations on sorted linked
lists. Rebalancing height-balanced trees is challenging while supporting
concurrent operations, and hence skip lists are used as a simpler alternative to
a deterministically balanced tree. A skip list is organized as a hierarchy of
linked lists, where higher-level lists are always contained in lower-level
lists. The list at the lowest level ($L_{0}$) contains the data elements in a
sorted order. We refer to the lowest level $L_{0}$ as the data layer, and the
upper levels as the index layers. Figure~\ref{fig:skiplist-design} shows the
structure of a skip list traditionally used in CPUs. A skip list employs skip
pointers for efficiently searching through the data structure (denoted by dense
black arrows in Figure~\ref{fig:skiplist-design}). Unlike balanced trees, a skip
list does not require expensive rebalancing after mutation operations while
achieving an expected logarithmic search time. Xing \etal survey different
variants of skip lists~\cite{skiplist-survey-2025}.

Misra and Chaudhuri~\cite{lockfree-ds-icpads-2012} ported the linearizable
construction of a lock-free skip list algorithm described by Herlihy and
Shavit~\cite{art-of-multiprocessor-programming-v2} to GPUs. Moscovici
\etal~\cite{gfsl-pact-2017} proposed GPU-friendly skip list (GFSL) which tries
to reduce the memory footprint and performance penalty due to uncoalesced global
memory accesses on a GPU. GFSL also uses the concept of super-nodes for linked
lists at each level. Figure~\ref{fig:gfsl-node} shows the organization of a
super-node in the index and data layers in GFSL. Each super-node in GFSL
contains $N + 2$ (key, value) pairs. A value field in an index layer $i$
(\code{DownPtr}) stores a pointer to the relevant super-node in the layer $i-i$,
and stores the data associated with the corresponding key in the data layer. The
last two pairs are used to store the maximum key in the node (\code{MaxKey}), a
pointer to the next node at the same level (\code{NextPtr}), and a lock to
ensure mutual exclusion during concurrent mutations.
Figure~\ref{fig:gfsl-design} shows a schematic of the GFSL design where each
super-node accommodates two keys. The blue dashed arrow shows the path taken
while searching for the key 11. A super-node is split if an insert operation
finds the node to be full, and a new super-node is added to the skip list
containing half of the values from the split parent super-node. In contrast to
traditional skip list, a key in GFSL is raised to the higher level only on a
node split operation. Like hash tables, the super-node is sized to match the
size of a warp (\ie, $N + 2$ is 32).

\begin{figure}[ht]
    \centering
    \includegraphics[width=\columnwidth]{"./figs/gfsl-node-overview1.drawio.pdf"}
    \caption{Super-nodes in index and data layers in GFSL}
    \label{fig:gfsl-node}
\end{figure}

\begin{figure}[ht]
    \centering
    \includegraphics[width=0.9\columnwidth]{"./figs/gfsl-design-overview-sb.drawio.pdf"}
    \caption{GFSL with two keys per super-node
      % and search of key 11
    }
    \label{fig:gfsl-design}
  \end{figure}

\section{Relevant GPU Hashtables}

As mentioned hashtables are data structures that stores key-value pairs. Now when two different keys collide for the same empty slot, hashtable provides two types of methods to solve this collision, which in turn divides hashtables in two broader classes: open addressing hashtable and closed
addressing hashtable.
Based on this two method, as far as our knowledge the most two relevant works are ~\cite{slabhash-ipdps-2020} and ~\cite{warpcore-hipc-2020}.
Slabhash uses a closed addressing based collision resolution technique. It uses a Work Coperative Work Sharing(WCWS) strategy, where each warp gets a(a block of 32 threads working in lock step manner) gets each operation. They uses specialized nodes(called slabnodes) that stores 31 key-value pairs and a pointer to next node. They uses a special memory allocator called slab allocator, which allocates a new slabnode whenever necessary.
\begin{figure}
    \centering
    \includegraphics[scale=0.7]{"./figs/slabhash.drawio.pdf"}
    \caption{Slabhash layout}
    \label{fig:slabhash-design}
\end{figure}

However, closed-addressing hash tables suffer from frequent random memory accesses. This behavior significantly degrades cache locality and renders hardware prefetching largely ineffective for large workloads. In particular, a key that would logically be placed in an adjacent slot may instead require the allocation of a new node located in a distant memory region, further exacerbating memory latency and performance variability.

Accordingly, we adopt WarpCore as the baseline for our design, which employs open addressing as its collision-resolution strategy. Open addressing offers superior cache locality, making it particularly well suited to our approach, as hardware prefetchers can be leveraged effectively. WarpCore also employs the Warp-Cooperative Work Sharing (WCWS) strategy: if a thread within a warp fails to insert a key, another thread in the same warp that has identified an empty slot attempts the insertion. If all threads in the warp fail, each thread computes a secondary hash and collectively probes the next consecutive 32 slots in the hash table.

Both the hashtable uses cooperative groups(cg) which is similar to a warp except now the warps becomes smaller with 2,4,8 or 16 threads(as specified in the program).

\begin{figure}
    \centering
    \includegraphics[scale=0.7]{"./figs/WCWS.pdf"}
    \caption{Warpcore layout}
    \label{fig:warpcore-design}
\end{figure}

But all of this works develop in memory hashtable.

\subsection{Manual Memory Management}

For most of such problems, the best way currently in usage is to execute the huge input workload in multiple batches such that each workload fits within the GPU memory. For certain cases, where a system has multiple GPUs ~\cite{warpdrive} proposed a multi gpu hashtable that can use both the GPUs and in turn can handle large workloads. But this type of setup is not only expensive but will also run into the same problem for even larger workloads.

\section{Designing UVM-based Data Structures}

As is evident from the discussion in the previous section, none of the existing
hash table and skip list design proposals consider GPU memory oversubscription.
As a first step toward enabling memory oversubscription in these data
structures, we discuss UVM-based implementations of hash table and skip list
that we use as the baseline for this work. We also point out the bottlenecks
faced by these UVM-based data structure designs and why existing techniques that
aim at optimizing UVM features fail to address them.

\subsection{Hash Tables with UVM Support}

Our hash table implementation extends WarpCore’s design [28] to support the
UVM programming model. Henceforth, we will refer to our UVM-enabled baseline
hash table design as HT-UVM. HT-UVM’s hash table is static, similar to
WarpCore. A limitation of a static design is that it requires an estimate of
the maximum number of insert operations to avoid expensive resizing at run
time. Hence, HT-UVM conservatively allocates an array of size
$\lceil (NUM\_INS\_OPS/LOAD\_FACTOR) \rceil$ in global memory using managed memory allocation
of CUDA, where NUM\_INS\_OPS is the number of insert operations to be
performed on the hash table and LOAD\_FACTOR indicates the fraction of filled
entries. HT-UVM implements three kernels: insert(), delete(), and search().
In addition to a pointer to the hash table in GPU global memory and an array
of input keys to be operated on, each kernel takes additional arguments. The
insert() kernel also takes as input an auxiliary array of values
corresponding to the input keys that are to be inserted, while the delete()
and search() kernels pass an output array as parameter which tracks the
values (corresponding to the input keys) that were operated on. The
operations in HT-UVM can be invoked both for a single key or for a batch of
keys; the latter will avoid the overhead of repeated kernel launches. HT-UVM
assigns each operation to a cooperative group (CG) which is commonly
referred to as warp-cooperative work scheduling (WCWS), and has been used in
prior work [6, 28].

Collision Resolution. HT-UVM uses open addressing with double hashing for
collision resolution because it can avoid the problem of clustering. Figure
5 shows how collisions are detected and resolved. The figure shows a CG
having four threads attempting to insert a key x with value y. Suppose the
key x hashes to a slot denoted by h 1 (x) in the figure, but the slot is
already occupied by key Ki . The other threads in the CG will linearly probe
the remaining slots after K i parallelly to find an empty slot where the
tuple (x, y) can be inserted. If all the slots probed by a CG are occupied
(indicated by the yellow slots), the CG probes alternate locations in the
hash table (indicated by the slots K j to K j+3 ) depending on the secondary
hash function h2 (x) used by double hashing. Threads with ranks 0 and 1 in
the CG find slots K j and K j+1 occupied, while the threads with ranks 2 and
3 find slots K j+2 and K j+3 unoccupied. The CG chooses the earliest
available slot K j+2 as the index for inserting x, and the corresponding CG
thread attempts to atomically store (x, y) in slot K j+2 . The insert
operation completes on a successful atomic store. A failed attempt indicates
a concurrent insert to K j+2, and the CG reattempts insertion starting from
the next empty slot K j+3 .

To enable a vanilla UVM support, we incorporate managed memory allocation for
the hash table array, i.e., replaced calls to cudaMalloc() and cudaMemcpy() with cudaMallocManaged().
The vanilla UVM implementation described thus far performs very poorly with more than 50%
oversubscription. A general good-quality hash function distributes the memory accesses across
different buckets, which leads to touching random pages depending on the input key pattern.
Oversubscription further leads to many pages storing hash table data to get evicted (based on
variants of LRU replacement policy implemented by the contemporary NVIDIA GPUs). This gives
rise to frequent far faults. Far faults are expensive because they incur overheads of page unmapping
on the CPU, page migration to the GPU, and installing page table entries in the GPU. Multiple
concurrent warps (or CGs) stall waiting for their respective far faults to be resolved, thereby hurting
performance [58, 59]. For example, in our experiments (details discussed in later sections), the
insert() kernel does not finish within even 24 hours with 100% oversubscription5 on a GPU with
16 GB memory.
We optimize the data movement in the vanilla UVM implementation by incorporating different
combinations of memory advise and asynchronous prefetch hints [51]. The best hint combina-
tion found through empirical evaluation turns out to be cudaMemAdviseSetAccessedBy for the hash
table array, and both cudaMemAdviseSetAccessedBy and cudaMemPrefetchAsync for the input array
containing keys to be operated on as well as the auxiliary array of values used by the insert()
kernel. The first hint conveys that the data structure is going to be predominantly accessed by
the GPU, allowing the runtime to pre-map the page table entries on the GPU without actually
migrating the data. The cudaMemPrefetchAsync hint allows the runtime to asynchronously prefetch
the hinted pages before the kernel starts, thereby reducing the initial far faults. We also experi-
mented with the cudaMemAdviseSetPreferredLocation and cudaMemAdviseSetReadMostly hints. The
cudaMemAdviseSetPreferredLocation hint indicates a preferred location (CPU or GPU) for a mem-
ory region, but it does not improve performance because the hash table pages anyway get evicted
due to oversubscription. The cudaMemAdviseSetReadMostly hint is not applicable as we update the
hash table across kernel launches.
We extend the warpcore design to cudaMallocManaged version for our baseline uvm as a result
our insert and search functionalities follows the similar patterns like warpcore. The core insert
function takes a key and a value and the pointer to the hashtable and uses warp intrinsics to find
an empty slot to insert the key and the value. In case of a duplicate, the value of that key is updated.
For search, similar lookup logic like insert is followed, but instead of a atomic instruction to insert
a key, search just returns the value associated to the particular key.

\subsection{Skip Lists with UVM Support}

We extend the GFSL [44] algorithm with UVM support to enable oversubscribing GPU memory. We
refer to this UVM-enabled skip list design as SL-UVM. SL-UVM supports insert(), delete(), and
search() operations, similar to HT-UVM. In the following, we first briefly describe the important
operations of SL-UVM, based on the GFSL proposal. Next, we discuss how we enable a vanilla
UVM support on top of GFSL.

UVM Support. To enable a vanilla UVM support, we incorporate managed memory allocation for
the super-nodes of GFSL. We also experiment with the memory advise and prefetching hints, and
find that only prefetch hints are useful for the super-nodes of SL-UVM and the input key array.
We do not use any hints for the auxiliary value array.

\subsection{Shortcomings of the Vanilla UVM Design}

To study the baseline UVM designs, we evaluate how the insert() and search() kernels of the
baseline hash table (HT-UVM) and the skip list (SL-UVM) scale with varying degrees of oversub-
scription. Figures 6a and 6b show the performance result for an input sequence that has unique
keys randomly distributed over the allowed range of 32-bit keys (such an input will be referred to
as a sparse unique input). The x-axis shows the length of the input key sequence that is operated
on, with 2e9 and 1.25e9 keys roughly saturating the GPU global memory for the HT-UVM and
SL-UVM designs respectively. Beyond 2e9 keys, the GPU memory is oversubscribed for HT-UVM
and the 100% oversubscription point is reached with 4e9 keys (ignoring the auxiliary value array).
On the other hand, for SL-UVM, the GPU memory is oversubscribed beyond 1.25e9 keys and the
100% oversubscription point is reached with 2.5e9 keys. In general, for the same input size, SL-UVM
has a bigger memory footprint from maintaining the down pointers between the skip list levels.

Figure 6a shows that the run time of the insert() kernel for HT-UVM increases steeply once the
working set starts oversubscribing the GPU memory. The run time of the search() kernel, on the
other hand, scales relatively slowly, but continues to exhibit a faster-than-linear growth beyond
2e9 keys. Figure 6b shows a similar trend for SL-UVM with search() being more expensive than
insert() up to 2e9 keys, beyond which point the insert() kernel ends up spending most of its time
waiting for far faults to resolve and fails to complete within 72 hours, which we set as a time-out
limit for all of our experiments with skip lists. This is indicated by a ‘×’ for the insert() kernel
when input lengths are 2.25e9 and 2.5e9. Importantly, the difficulty of scaling a skip list is also
reflected in the timescale of the y-axis of Figure 6b which is in hours compared to that in seconds
for the hash table.
The difference between the insert() and search() kernels are that the latter works on a pre-built
data structure. The search() kernel is slower than the insert() kernel for the skip list (Figure 6b)
because it operates on a prebuilt skip list and has potentially poorer locality compared to inserts.
With oversubscription, the search() kernel suffer from poor locality from the very beginning
because the search key could lie anywhere in the skip list, while the insert() kernel enjoys
comparatively better locality and does not encounter far faults until the skip list starts overflowing
GPU memory.
The primary reason for poor scalability of HT-UVM and SL-UVM is the lack of any locality in
the input key pattern leading to accesses going to distant parts of the data structure within a short
time window. With oversubscription, this increases the possibility of the requested page being not
resident in the GPU memory hierarchy, leading to frequent far faults.

To further study the impact of memory advise hints on the page fault volume, we evaluate the
insert() kernel of HT-UVM with and without these hints using an input sequence that completes
within a reasonable amount of time when hints are disabled. This input sequence uses monotonically
increasing unique keys, thereby inducing spatial locality in the access pattern.6 Figure 7a shows the
volume of far faults observed by the insert() kernel of HT-UVM on a monotonically increasing
input without any hint, while Figure 7b shows the volume of remote mappings (a reasonable proxy
for far faults) when hints are enabled. We observe that the number of page faults is significantly
reduced with memory hints. The hint cudaMemAdviseSetAccessedBy, used for the hash table pages,

does not cause data migration, but only sets up virtual to physical address translations in the
GPU [51]. Instead of evicting the existing pages from GPU memory and migrating pages, this
hint allows the GPU to access the hash table pages from the CPU’s system memory through the
communication network (e.g., PCIe or NVLink). These accesses lead to remote mapping events.
This is why the volume of remote mappings captures the overhead of accessing the hash table
pages from the system memory. However, remote mappings help only when the degree of GPU
memory oversubscription is small because remote accesses are faster than serving frequent page
faults by the driver. As the input sequence length increases, the increasing degree of memory
oversubscription results in a superpolynomial increase in the volume of remote mappings, as can be
seen in Figure 7b. This, in turn, increases the communication and contention on the PCIe/NVLink
leading to overall poor scalability of the HT-UVM design even in the presence of the memory
advise hints. As a result, frequent use of remote mappings may result in poorer performance than
page fault serving in certain situations. Thus, the optimal memory advise hints for a data structure
may depend on the input pattern as well as the input sequence length, adding an extra layer of
complexity for the programmer and static program analyses [10].

Existing techniques, such as prefetching and page replacement policies, optimize memory usage,
but often fail to be effective when applications with irregular access patterns oversubscribe GPU
memory [3, 5, 19, 67, 68]. Applications with irregular access patterns, e.g., graph-based applications
and concurrent GPU data structures that require pointer chasing, have poor locality, making
prefetching counterproductive. Thus, it is imperative to come up with targeted novel data structure
designs that can scale with the input size.

\section{Problem Statement and Proposed Approach}

Scaling HT-UVM and SL-UVM to high degrees of GPU memory oversubscription requires novel
designs that take into consideration the potential lack of locality in the access pattern. In the
following, we discuss the design of a hash table and a skip list that scales better with input size.


We have designed UVM-based implementation of hash tables with modified structures that performes well even with memory oversubscription. We compared our design with the existing state-of-the-art hashtables like cucollections\href{https://github.com/NVIDIA/cuCollections/tree/dev} and ~\cite{warpcore-hipc-2020}. For valid comparison, we update this models to support memory oversubscription and we observe that they suffer greatly because of far faults. So we designed our hashtable such that these far faults are reduced.
Our current work is currently under submission in a reputed conference.

\subsection{Designing Oversubscribed Data Structures}

The key insights for minimizing page faults are to (i) group nearby keys in the input sequence
to enhance the locality of the input, and (ii) translate the locality, if any, present in the input
sequence to the hash table architecture. The first requirement ensures that keys within the same
page are processed together (i.e., near in time) leveraging spatial locality. The second requirement
implies that the nearby keys map to nearby memory pages of the hash table. This is, in general,
challenging because a general hash function may end up mapping even consecutive keys to far
apart pages. Taken together, the aforementioned two requirements optimize memory access pattern
and facilitate efficient prefetching, which, in turn, reduces the frequency of on-demand page faults
and subsequent stalls during operations improving overall performance

\section{Ongoing Work}

We are planning on extending this idea to amq ie, designing a amq that doesn't suffer a huge performance overhead for memory oversubscription. We are
also planning on improving the algorithms of the current state-of-the-art amqs such that they receive a substantial performance boost.

\bigskip
\noindent In summary, our proposal aims to bridge the gap between theoretical
virtual memory management optimizations and practical implementations, enabling
more efficient handling of oversubscription and irregular access scenarios in
future GPU architectures.

{\small
\bibliographystyle{plain}
\bibliography{./references/venue-full,./references/references}
}

\end{document}
